\Titolone{Ringraziamenti}{\noindent\hfill ``$\dots$ Forse se nascevo in un posto migliore ero un genio, un avvocato, un inventore $\dots$''}{Paolino Paperino Band-Pislas}
Iniziamo a comporre la sequenza di parole che risulta la vera ragione di questa tesi (se non di tutta la mia carriera accademica): i ringraziamenti. Prima di tutto ringrazio tutti quelli che nell'arco di questi $6$ anni hanno trascorso tempo con me, divertendoci assieme, a tutti quelli che mi hanno invitato a cena o mi hanno fatto favori speciali, tutte queste pagine finali sono per voi (no, la bibliografia non \`e per voi) per ringraziarvi di cuore e se il vostro nome non lo trovate, sar\`a la vecchiaia oppure il desiderio di non offendere nessuno.
\par
Inizio subito nel ringraziare gli elementi umani che hanno permesso a livello amministrativo questa laurea e non solo per firme e regole fonetiche: i ``ragazzi'' di Palazzo Campana ed in particolare Lorenzo Fatibene, Mauro Francaviglia e Gianluca Allemandi (e tutti quelli che leggendo il loro libro si chiederanno ``ma chi \`e Gianluca Pacchiella'').
\par
Dopo i saluti amministrativi di rito, partiamo con il delirio: saluto il mio fratellinio accademico cio\`e  il Roberto Lia nazionale \Foot{Si legge Ro-ber-to Lia-a-a-a-a-a.}, uomo dotato di orizzonte degli eventi causale ma dotato di un cuore non limitato superiormente, ringrazio anche la sua famiglia oltre per gli eventuali favori anche per il lavoro fantastico che hanno fatto con lui. Passiamo quindi al fratello maggiore: Gianni, ``masterizzatore'' a Milano e che devo ringraziare un miliardo di volte per le serate passate insieme, per le domande fatte  e quelle non fatte, per il viaggio in Olanda e per le altre cose di cui ormai non mi ricordo pi\'u.
\par
Volendo passare ai cugini, saluto Stefano Cimito\Foot{Ovviamente salutami Marco e gli altri tuoi amici; non sapevo come inserirli.}\ polivalente uomo capace di passare da una sfolgorante carriera ciclistica ad una di menestrello professionista, grazie per tutte le sue domande su cose relative ai circuiti nella fiducia che io ne sapessi qualcosa, spero che a Santiago ti diano la condizionale; saluto Marco Allasia il tuttologo di fama internazionale, colpito anche lui dalla sindrome da stambecco, con cui mi scuso per le volte che esibendomi in frasi di indubbia moralit\`a l'ho costretto a strisciare il suo corpo sulle tastiere della sala informatica, lo ringrazio inoltre per la passione che infonde nella degustazioni di liquori, passione del resto comune a molti abitanti del palazzo di fisica: in particolare Elio, grazie al quale \`e stato possibile scrivere con tranquillita la tesi in questo ambiguo agosto\Foot{Non dimentico certo l'anguria insaporita nel giorno del mio compleanno.}\ e al mio ex ``padrone di casa'' Enrico, capace di comprendere ogni porzione di quell'arte arcana chiamata informatica.
\par
A proposito di case, saluto quei due pasticcieri che mi hanno ospitato senza tanti problemi e che hanno permesso la stesura della tesi senza problemi ulteriori:   Leandro (a cui dedico sempre quella mitica frase di Bukoski che spero lui si ricordi\Foot{Uuuuhhhhh...ricordi.}) e Dello, che ringrazio per le serate passate a discutere le strane conseguenze delle leggi fisiche.
\par
Non dimentichiamoci del nostro ``games administrator'' il pierpi, ragazzo con cui ho condiviso molte esperienze (W il $G8$) e con cui mi devo ancora scusare per come mi sono comportato al mio compleanno; il galateo prevede anche un saluto alla Rita, donna che ha capito tutto essendosi messa insieme al precedentemente citato.
\par
Villa Palmeri va ringraziata perch\'e \`e Villa Palmeri: i suoi abitanti passati e futuri, Fabio per non essersi stufato di vedermi dopo $8$ mesi che abbiamo vissuto assieme, Vanira per avermi fatto conoscere il magico mondo del Sud; saluto Checco e tutta l'altra banda del Salento, grazie per l'invito di questa estate ma mi avete capito.
\par
Del mio anno accademico, saluto quel napoli di Jerry, quel barese di Zeverino (ma dove sei sparito?), quel comico di Mattia\Foot{Ci siamo divertiti eh?} e quel pazzo di Squatter; un saluto particolare a Geova: alla sua parte reale (gran signore) Pino Vallone e alla sua parte immaginaria denominata in campo reale Simone Giacometti, uomo capace di affettare salumi e pensare alle misure boreliane, ti meriti sicuramente pi\'u di me la laurea.
\par 
Potrei non salutare Pastis? Ovviamente no e allora lo ringrazio per le sconvolgenti rivelazioni sul capo della chiesa cristiana, le serate da Giancarlo e le fritture di pesce e per l'accoglienza al ritorno dall'Olanda; saluto Anna e Ivan perch\'e sono Anna e Ivan, grazie per questi anni passati assieme nel bene e nel male (in particolare per tutti i regali fatti a natale e pasqua da Pagnotta).
\par
Saluto quei tracotanti dei Mirafiorensi amici di quel Roby Lia: l'unbelievable Della, il bellissimo Vigno, l'Alessio e lo Stoja; un saluto particolare a Fiabetto (che l'inverno ha colto ad Aprile), a Losque (grazie veramente per tutte le serate passate assieme), all'enumerabile Bobo che ci ha insegnato che il numero perfetto \`e possibile. Ovviamente plaudo a quei mostri chiamati Teo\Foot{Aprofitto per salutare i suoi compagni di Agraria (Emi e Pavia) oltre che pupino}\ e Isac, capaci di tutto se messi nelle giuste condizioni.
\par 
Un saluto sparso a tutti quelli che fanno parte degli  anni dopo il mio: al pirata Ciuck, al sionista Masoero, al riccioluto Matteo, quel metallaro di Lazzarini e a quel pazzo di Monti, senza porre in disparte la giavellottista  Isa e la ragazza capace di dominare le farfalle nello spettro visisbile: Chicca (colgo l'occasione per salutare l'uomo che lei domina, cio\`e Paolo di Cuneo e l'altro compagno di corso di quest'ultimo, Andrea Elmi). Ringrazio Matteo Vallerossa, incredibile elemento a cui non si pu\`o non voler bene e saluto le grazie di fisica: Giovanna (che sarebbe meglio non si pigliasse pi\'u i funghi), Viviana, Nadia e Francesca Moscat.
\par
Non pensavo di arrivare a questo punto senza salutare la mitica regina di cuori Anna: grazie veramente perch\`e sei l'unica che riesce a non farmi cadere in un maschilismo disperato, saluto Pedro, che mi sta guardando mentre scrivo queste righe per tutti i film a cui ha fatto parte, la lasciva di Simona, a quel genio senza limiti di Stefano Lottini, quel punkabbestia di Martino Gagliardi e quel giovane matematico di Federico a cui spero le macchine non facciano pi\'u scherzi dal punto di vista termodinamico. Mi stavo dimenticando di Vibl (laureando stamattina) capace di un'accoglienza senza limiti, Domenico, unico semidio in terra ancora costretto con un grosso fardello per l'umanit\`a e l'unico, l'inimitabile Sciretti, dottorando a Saragozza, capace di tornare dall'erasmus e dormire in macchina quando doveva accompagnarmi a casa. Saluto Rocco, Bari$84$ e Famous\Foot{No Leociao no.}\ oltre che l'uomo in orbita su Saturno in un polmone d'acciaio. Saluto i ragazzi del (ex) primo anno: Sticks, Jianlu, John, Arianna, Carla e gli altri che non mi ricordo neanche pi\'u.
\par
Ringrazio i ragazzi che hanno fatto parte della mia vita pre-universitaria: il mio ex gruppo a partire da Beppe (ti tatuerai anche il mio nome un giorno), a Dario a cui dico che adesso sono io quello che tiene duro (ma con chi?) e al mitico Ciccio, laureato ma pieno di problemi. Saluto anche i chitarristi: Mauro e Pelle e tutti gli altri componenti della Torino Hardcore: Calle, Alice e figlio, i ``banda del Rione'', gli ``Younggang'', i fratelli Siragusa tutti (oltre che la famiglia), Manu ed il discendente dal nome irricordabile; un saluto particolare a Jessica, unica ragazza capace di fiaccarmi dal punto di vista etilico (quella sera i mirafiorensi se la ricordano bene).
\par
\bigskip
Ultimo ma sicuramente non meno importante \`e il ringraziamento alla persona a cui dedico la tesi: a mia madre che oltre ad avermi  permesso di  nascere, mi ha insegnato le cose pi\'u importanti della vita senza usare parole; grazie per tutto quello che hai fatto per me, che non ho saputo e che non sapr\`o mai e per le quali ormai non ti posso pi\'u ringraziare.
\bigskip
\line{\hfill$\hbar$\hskip 1cm}
