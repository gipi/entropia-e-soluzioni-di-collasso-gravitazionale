%
% File di macro per tesi
%
%%%%%%%%%%%%%%%%%%%%%%%%%%%%%%%%%%%%%%%%%%%%%%%%%%%%%%%%%%%%%%%%
% Definizioni Font
%%%%%%%%%%%%%%%%%%%%%%%%%%%%%%%%%%%%%%%%%%%%%%%%%%%%%%%%%%%%%%%%
% Li puoi trovare nella directory: /usr/share/doc/texmf/fonts/ Per la Debian
% /usr/share/texmf/fonts/tfm/public/pxfonts/
%%%%%%%%%%%%%%%%%%%%%%%%%%%%%%%%%%%%%%%%%%%%%%%%%%%%%%%%%%%%%%%%
% Definisco il font da usare nei titoli capitolo
%\font\tenrm=pxr % Palatino
\font\Citazioni=p1xsc at 7pt
\font\WWW=p1xsc at 10pt
\font\VecchioStile=pcxr at 5pt %Scrive (non in modo Math) cifre tipo \oldstyle
\font\Fronte=p1xsc at 20pt
%\def\pal{\let\rm=\tenrm \baselineskip=12.5pt \rm}
\font\titoloCap=pxr at 25pt
\font\titoloCapUno=pxb at 30pt
\font\titoloSec=pxr at 20pt
\font\titoloSymbolSec=pxbsy at 20pt
\font\SottoSezioneFont=pxb at 14pt

%Qui carichiamo i font per il testo normale
\def\Text{pxr }
\font\PalatinoTextTen=\Text at 10pt
\font\PalatinoTextSeven=\Text at 7pt
\font\PalatinoTextFive=\Text at 5pt
\textfont0=\PalatinoTextTen
\scriptfont0=\PalatinoTextSeven
\scriptscriptfont0=\PalatinoTextFive 
\def\rm{\fam0\PalatinoTextTen}%
% Font Grassetto
\def\Bold{pxb }
\font\PalatinoBold=\Bold at 10pt
\def\bf{\fam\bffam\PalatinoBold}
\def\pal{\rm \baselineskip=12.5pt}
%Font italici
\def\It{pxi }
\def\Italic{\PalatinoItalicTen}
\font\PalatinoItalicTen=\It at 10pt
\textfont\itfam=\PalatinoItalicTen
\def\it{\fam\itfam\PalatinoItalicTen}
% Qui carica i Font Matematici
\def\Math{pxmi }
\font\PalatinoMathTen=\Math     at 10pt
\font\PalatinoMathSeven=\Math   at 7pt
\font\PalatinoMathFive=\Math    at 5pt
\textfont1=\PalatinoMathTen
\scriptfont1=\PalatinoMathSeven
\scriptscriptfont1=\PalatinoMathFive
% Qui carica i font per i simboli
\def\MathSymbol{pxsy } 
\font\PalatinoMathSymbolTen=\MathSymbol at 10pt
\font\PalatinoMathSymbolSeven=\MathSymbol at 7pt
\font\PalatinoMathSymbolFive=\MathSymbol at 5pt
\textfont2=\PalatinoMathSymbolTen
\scriptfont2=\PalatinoMathSymbolSeven
\scriptscriptfont2=\PalatinoMathSymbolFive


% Bold Matematico
\font\PalatinoMathBold=pxbmi at 10pt
\font\PalatinoMathBoldSeven=pxbmi at 7pt
\textfont\bffam=\PalatinoMathBold
\scriptfont\bffam=\PalatinoMathBoldSeven 
\scriptscriptfont\bffam=\PalatinoMathBold %

\def\MathMoreSymbol{pxex } 
\font\PalatinoMathMoreSymbolTen=\MathMoreSymbol at 10pt
\font\PalatinoMathMoreSymbolSeven=\MathMoreSymbol at 7pt
\font\PalatinoMathMoreSymbolFive=\MathMoreSymbol at 5pt
\textfont3=\PalatinoMathMoreSymbolTen
\scriptfont3=\PalatinoMathMoreSymbolSeven
\scriptscriptfont3=\PalatinoMathMoreSymbolFive

\textfont\itfam=\PalatinoTextTen \def\it{\fam\itfam\PalatinoTextTen}%

\font\Simbolo=pxsy at 20pt\newfam\Famiglia\textfont\Famiglia=\Simbolo %<-- Simbolo capitolo alla landau
\font\SimboliSgamo=pxsya at 10pt
\newfam\Boxies\textfont\Boxies=\SimboliSgamo
\def\Box{\mathchar"\the\Boxies03}
\def\BBox{\mathchar"\the\Boxies04}
%
\font\Gotico=pxmia%<--- Definisce il Gotico
\font\Gruppi=pxsyb%<--- Definisce il carattere ''sbarretta''
% Per le didascalie
\font\CaptTitle=pxb at 7pt
\newfam\CaptFam
\textfont\CaptFam=\PalatinoMathSeven
\def\Caption{\fam\CaptFam\PalatinoTextSeven}
%%%%%%%%%%%%%%%%%%%%%%%%%%%%%%%%%%%%%%%
% Parametri della pagina
\hoffset 1cm
%\voffset 3.5cm
\hsize 13cm
%%%%%%%%%%%%%%%%%%%%%%%%%%%%%%%%%%%%%%
% Togliamo fastidiose sbarre nere
\overfullrule 0pt
%%%%%%%%%%%%%%%%%%%%%%%%%%%%%%%%%%%%%
% Hyphenation italiano
%\language4
%%%%%%%%%%%%%%%%%%%%%%%%%%%%%%%%%%%%%%
% Definisco i contatori che mi sono utili
\newcount\chapN  %<--- Per i capitoli
\newcount\secN   %<--- Per le sezioni
\newcount\appN   %<--- Per l'appendice
\newcount\equatN %<--- Per le equazioni
\newcount\footN  %<--- Per le note a pie' pagina
\newcount\figN   %<--- Per le figure
\newcount\defN   %<--- Per le definizioni
\newtoks\ChapMark%<--- Per il nome del capitolo corrente

%%%%%%%%%%%%%%%%%%%%%%%%%%%%%%%%%%%%%%%%%%%%%%%%%%%%%%%%%%%%%%%%%%%%%%%%
\def\Vec#1{{\bf #1 }}
\def\Vers#1{{\bf e}_{\hat{#1}}}
\def\tensor#1{#1\otimes #1}
\def\Mink{%
Min\-ko\-wski%
}
\def\Sch{%
Schwarz\-schild%%%%%
}
\def\ST{%
spazio-tempo%
}
\def\Mat#1#2#3{%<--- Macro per indici diagonali nei tensori
{#1}^{#2}_{\phantom{#2}{#3}}
}
\def\DC#1{\nabla_{#1}}
\def\Slash#1{#1\llap{/}}
\def\etaM{\eta\lower 1.5ex\hbox{$\scriptscriptstyle (M)$}}
\def\gSch{g{\lower 1ex\hbox{$\scriptstyle Sch$}} }
\def\Oe{oriz\-zon\-te de\-gli even\-ti}
\def\SM{% <--- Macro per scrivere il simbolo maiuscolo
\lower 1.5pt\hbox{$\mathchar"\the\Famiglia78$}
}
\def\MM{%<--- Scrive il simbolo di varieta'
{\cal M}
}
\def\RN#1{%<--- Macro per R^#1
\hbox{\Gruppi R}^{#1}
}
\def\PT#1{%<--- Macro per base TM
\partial_{#1}
}
\def\chart#1#2#3{%<--- Per le funzioni coordinate Punto Indici Carta
\smash{\buildrel{\scriptscriptstyle(#3)}\over {#1}}\vphantom{#1}#2
}
\def\Item#1{%<--- Per i miei item
\item{$\triangleright$}{\bf #1}%
}
\def\Help#1{%<--- Serve per i commenti da parte dell'autore
%({\bf ?})%
\write\Aiuto{\Item{?}#1 $\to$ pg.\folio\smallskip}%
}
\def\Lie#1#2{%<--- Derivata di Lie
{\cal L}_{\scriptscriptstyle#2}{#1}
}
\def\GC#1#2#3{%<--- Macro per le Gamma Christoffel
\Gamma^{#1}_{\phantom{#1}{#2}{#3}}
}
\def\Rs{% <--- Macro per il termine della metrica di \Sch
1-{2M\over r}
}
\def\Tensor#1{%<--- Macro per dx(X)dx
#1\otimes #1%
}
\def\Oriz{%<--- Macro per H+
${\cal H}_{+}$%
}
% MISCELLANEA
\def\vip#1{{\bf #1}}          %<--- Per le cose importanti
\def\FinePagina{\vfill\eject} %<--- Passa ad un'altra pagina
\def\Figura#1#2#3#4{%<--- Macro per inserire figure centrate dando le dimensioni oltre che mettere le didascalie
\midinsert
\par\vbox{\bigskip\line{
\hfill\vbox to #3{
\hsize #2\vfill\special{psfile=#1.eps}
}
\hskip #2\hfill 
}\smallskip\global\advance\figN by 1
\line{\hfill\vtop{{\CaptTitle\noindent Figura \the\figN:}\hsize #2\Caption #4}\hfill}
\bigskip}
\endinsert
}
\def\Def#1#2FineDef{ % <--- Definizioni
\medskip\global\advance\defN by 1
{\bf Definizione\the\chapN.\the\defN: #1}
\par\noindent
%\line{\hfill\vbox{\noindent\hsize 14cm #2}\hfill}
#2
\hbox{\hfill$\spadesuit$}
\smallskip
}
\def\Nota#1FineNota{%<--- Macro per i Nota Bene
\bigskip\noindent
{\bf Nota Bene:} #1
\bigskip
}
%%%%%%%%%%%%%%%%%%%%%%%%%%%%%%%%%%%%%%%%%%%%%%%%%%%%%%%%%%%%%%%%%
%Variabii Utili
%%%%%%%%%%%%%%%%%%%%%%%%%%%%%%%%%%%%%%%%%%%%%%%%%%%%%%%%%%%%%%%%%

%Definisce se e' una pagina di inizio capitolo
\newif\ifCapYes%
\CapYesfalse%
% devo settare il numero pagine?
\newif\ifNoPage
\NoPagefalse
%%%%%%%%%%%%%%%%%%%%%%%%%%%%%%%%%%%%%%%%%%%%%%%%%%%%%%%%%%%%%%%%%
% CAPITOLI, Sezioni, footnote, Table of Contents
%%%%%%%%%%%%%%%%%%%%%%%%%%%%%%%%%%%%%%%%%%%%%%%%%%%%%%%%%%%%%%%%%
% Capitolo
\def\Chapter#1#2#3{\global\secN=0 \FinePagina\CapYestrue % Nuova pagina e setta Inizio Capitolo=vero
\hbox{\titoloCap \global\ChapMark={#1}\global\advance\chapN by 1 \noindent Capitolo \the\chapN}
\bigskip
\noindent{\titoloCapUno #1}
\vskip  2cm
%\hbox{\hfill\vtop{\hsize 6cm\Citazioni #2}}
\hbox{\hfill\vtop{\leftskip = 1in %\rightskip =1in 
\noindent\Citazioni #2%
%\hfill + forte di \hfil
\smallskip {\hfill\Citazioni #3}}}
\vskip  2cm
\AggiungiVoce{\bigskip \hbox{%
\titoloCap\the\chapN \ /\vtop{#1 \hrulefill\folio}
}}
}

% Sezione
\def\Section#1{%
\vskip 2cm\global\advance\secN by 1 
\hbox{\vbox{\noindent\titoloSec \raise 3pt\hbox{\SM}%${\aleph}$} 
\the\secN\hskip 20pt #1}}
\nobreak\medskip
\AggiungiVoce{\medskip\indent\indent {\PalatinoBold #1 \dotfill pg.\folio}}
}
% Sezione Pacco
\def\SectionPacco#1{%
\par\bigskip%
\hbox{\vbox{\noindent\titoloSec #1}}
\bigskip}

\def\Titolone#1#2#3{
\FinePagina\CapYestrue
\line{\titoloCap\hfill}
\bigskip\global\ChapMark={#1}
\noindent{\titoloCapUno#1}
\vskip  2cm
%\hbox{\hfill\vtop{\hsize 6cm\Citazioni #2}}
\hbox{\hfill\vtop{\leftskip = 1in %\rightskip =1in 
\noindent\Citazioni #2%
%\hfill + forte di \hfil
\smallskip {\hfill\Citazioni #3}}}
\vskip  2cm
\AggiungiVoce{\bigskip \hbox{\vrule width 12pt depth 0pt\ %
\vtop{\titoloCap#1 \hfill\folio}
}}}
% SottoSezione
\def\SottoSezione#1{%
%\bigbreak
\vskip 1cm\noindent{{\bf #1}\hrulefill}%
\nobreak
\AggiungiVoce{\smallskip\indent\indent\indent #1 \dotfill pg.\folio\hskip 1em}
\bigskip\noindent
}
% Appendici
\def\Appendix#1{\FinePagina\CapYestrue
\hbox{\global\ChapMark={#1}\titoloCap \global\advance\appN by 1 \noindent Appendice \uppercase\expandafter{\romannumeral\the\appN}}
\bigskip\noindent{\titoloCapUno #1}\vskip 3cm%
\AggiungiVoce{{\titoloCap\bigskip\noindent\uppercase\expandafter{\romannumeral\the\appN} /\ #1\hfill\folio}}
}
% Esempi
\def\Example#1#2FineEx{%<--- Macro per gli esempi
\smallskip
\line{\hfill\vbox{\hrule height 1mm width 5cm}\hfill}
{\bf Esempio:} #1\smallskip
\noindent #2\smallskip
\line{\hfill\vbox{\hrule height 1mm width 5cm}\hfill}\smallskip
}
% Note a pie' di pagina
\def\Foot#1{%
\global\advance\footN by 1%
\footnote{${}^{\the\footN}$}{#1}%
}

% Headline
\headline={%
\ifCapYes% se e' la pagina iniziale capitolo la headline vuota
\hfill%
\global\CapYesfalse % se non vi e' il global non funge
\else%     se no fai una riga orizzontale
\vbox{{\hfill\the\ChapMark}\hrule height 0.3mm depth 0.1mm}% depth 0.1cm%
\fi%
%
}
\footline={%
\ifNoPage\hfill\global\NoPagefalse
\else
\hfill\folio\hfill
\fi
}
\def\COUNT#1{% <--- Macro per registrare numeri ``on fly'' si usa subito dopo l'equazione (NON FUNZIONA DENTRO L'AMBIENTE MATEMATICO!)
	\expandafter\edef\csname #1\endcsname {%
	(\number\the\chapN.\number\the\equatN)%
}}
%%%%%%%%%%%%%%%%%%%%%%%%%%%%%%%%%%%%%%%%%%
% Sommario, riferimenti e bibliografia
%%%%%%%%%%%%%%%%%%%%%%%%%%%%%%%%%%%%%%%%%%
% TOC
\def\Sommario{% <--- Macro per TOC
\FinePagina\CapYestrue
\vbox{\line{\global\ChapMark={Indice}\titoloCapUno Indice\hfill}%\hrule height .5mm depth .5mm
\noindent\vskip 3cm
}%
\vskip 1cm%
\input \jobname.toc %
}

\def\Biblio#1#2#3#4{% <--- Macro perla bibliografia
[#1]% Questo scrive anche sul testo
\write\Bib{\hbox{\hbox to 2cm{[#1]\hfill}\vtop{\noindent#2.``#3''\par{\Italic #4}}}\bigskip}%
}

% MATEMATICA
\def\Eqno{% <--- Macro per le equazioni
\global\advance\equatN by 1
\eqno{(\the\chapN.\the\equatN)}
}

% Per eventuali inserzioni LaTex generate da Maple
\def\frac#1#2{% <--- Inserzioni LaTeX generate da Maple 
#1\over #2
}
%%%%%%%%%%%%%%%%%%%%%%%%%%%%%%%%%%%%%%%%%%%%%%%%%%%%%%%%%%%%%%%%%%%
% Macro per l'indice
%%%%%%%%%%%%%%%%%%%%%%%%%%%%%%%%%%%%%%%%%%%%%%%%%%%%%%%%%%%%%%%%%%%

\bigskip
\def\AggiungiVoce#1{% <--- Macro che inserisce una voce nel file \jobname.toc
\write\TOC{#1}
}







