\Titolone{Introduzione}{{\bf Fisica:} la disciplina che ha per oggetto lo studio della natura, le cui caratteristiche e i cui metodi  sono pertanto in relazione con quello che si intende per natura.}{Dizionario della Filosofia UTET}\pageno=1
\language4
Dalla formalizzazione della teoria della \vip{Relativit\`a Generale}\ nel 1916 da parte di Einstein, si \`e assistito ad una completa rivoluzione delle leggi della fisica dovute sia all'avvento di questa nuova teoria sia a numerose  novit\`a introdotte dallo stesso Einstein e da altri scienziati del calibro di Born, Heisemberg e Schrondinger; molte novit\`a riguardano la {\vip meccanica quantistica} introdotta dopo pochi anni.
\par
La teoria della relativit\`a rimase praticamente inalterata (se non nelle tecniche di calcolo e in proposte di modifiche ancora mai accettate universalmente) fino al giorno d'oggi mentre i contributi relativi alla teoria quantistica furono rapidi e portarono nel giro di pochi anni alla creazione di una nuova teoria fisica: la teoria \vip {quantistica di campo}. In questa nuova teoria  le interazioni esistenti in natura sono mediate da particelle di spin intero che sono   interpretate  come eccitazioni di questo campo rispetto ad uno stato di vuoto; qusto formalismo ha avuto un successo insperato: la verifica sperimentale delle previsioni della conseguente \vip{elettrodinamica quantistica}\ a livelli di precisione  mai raggiunti prima  nella storia di tutta la fisica portarono con entusiasmo all'estensione di questa teoria a comprendere anche i fenomeni del decadimento $\beta$, introducendo una forza elettrodebole con una cosidetta simmetria di \vip{gauge}. Interessanti risultati sono stati  ottenuti anche dal tentativo di applicare le medesime tecniche alla forza forte, responsabile dei fenomeni nucleari.
\par
Questo processo ha lasciato quasi inalterato il lavoro di Einstein che a tutto oggi non ha trovato una sua posizione definitiva in questa teoria dei quanti: vari tentativi sono stati fatti in direzioni diametralmente opposte tuttavia nessuno di questi  ha  portato a risultati soddisfacenti (o almeno universalmente accettata); in una visione unificata dei fenomeni naturali questo risultava un grosso problema.
\par
Negli anni 70 il fisico S.W.Hawking, applicando la teoria dei campi ad un campo gravitazionale di background di un corpo in collasso, dimostr\`o che un osservatore molto lontano dal corpo nel ``futuro'', osserva un flusso costante di particelle secondo uno spettro di corpo nero avente una temperatura proporzionale al valore della forza gravitazionale misurata sulla superficie del corpo stesso. Nel frattempo era stata proposta una analogia formale fra le leggi della termodinamica e  alcune grandezze proprie dei buchi neri, lo stato finale del collasso di una stella: in particolare l'area della superficie di un buco nero era messa in relazione con l'entropia posseduta dal corpo stesso.
\par
L'importanza di questa analogia consisteva nella relazione tra una quantit\`a pu\-ra\-men\-te classica (l'area dell'orizzonte degli eventi) e una fondamentalmente quantistica (l'en\-tro\-pia). A livello della cosidetta matrice statistica \`e necessario conoscere tutti i cosidetti microstati che si possono individuare solo tramite una completa teoria quantistica. La possibilit\`a di calcolare in maniera indiretta questa grandezza da considerazioni macroscopiche, permette di fare previsioni sul comportamento di una teoria  quantistica della gravit\`a in un regime semiclassico.
\par
In questo lavoro si \`e cercato di raccogliere e rielaborare le conoscenza necessarie per interpretare geometricamente e macroscopicamente queste quantit\`a di origine quantistica (l'entropia e la temperatura). Inoltre abbiamo tentato di analizzare le situazioni limite che come noto accadono durante il collasso di un sistema di polveri. \`E noto che questi sistemi possono sviluppare singolarit\`a nude.\par
Questo potr\`a forse dare la possibilit\`a di interpretare questi oggetti dal punto di vista geometrico e fisico, ponendo forse un punto di partenza per la loro comprensione (che \`e attualmente parziale e lacunosa).