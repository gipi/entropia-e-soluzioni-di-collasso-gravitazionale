\FinePagina\CapYestrue\global\ChapMark={Notazioni}
\hbox{\vrule width 0.5mm\vbox to 5cm{\hbox{\titoloCapUno Notazioni}%\hrule depth .1cm
}\hfill}
\vskip 3cm
Visto la enorme diffusione di diversi tipi di notazioni qui riporter\'o quelle usate da me nel seguito del testo
\Item{} La metrica da noi usata in coordinate galileiane \`e $\eta_{\mu\nu}=diag(-,+,+,+)$; nel seguito del testo ci riferiremo ad essa in generale con $\etaM$.
\Item{}\ Con gli indici greci ($\mu,\nu,\alpha,\beta,\dots$) si indicano componenti di tensori aventi  un campo di definizione quadridimensionale, con gli indici latini($a,b,c\dots$) componenti tridimensionali; infine con indici incappucciati componenti nel ``rest frame''.
\Item{}\ I vettori tridimensionali con il simbolo usuale di vettore: $u=(u^0,\vec{u})$
\Item{}\ Se non sono presenti parentesi gli operatori differenziali agiscono solo sull'oggetto alla loro immediata destra. Gli operatori di derivata parziale sono indicati con $\partial_{\mu}={\partial\over \partial x^{\mu}}$.
\Item{}\ Si denota la simmetrizzazione e la  antisimmetrizzazione rispettivamente con 
$$
\eqalign{
&A_{\{\mu_1\dots\mu_n\}}=\sum_{\sigma} {1\over n!}A_{\sigma(\mu_1)\dots\sigma(\mu_n)}\cr
&A_{[\mu_1\dots\mu_n]}=\sum_{\sigma} {1\over n!}sign(\sigma)A_{\sigma(\mu_1)\dots\sigma(\mu_n)}
}
$$
dove $\sigma$ \`e una permutazione.
\Item{}\ Dove in una equazione tensoriale appaiono una coppia di indici ripetuti si sottintende una sommatoria:
$$
\Mat{\Omega}{\mu\nu}{\rho}=\Mat{\zeta}{\mu}{\alpha}\Mat{\Phi}{\nu\alpha}{\rho}=\sum_{\alpha=1}^n\Mat{\zeta}{\mu}{\alpha}\Mat{\Phi}{\nu\alpha}{\rho}
$$
dove $\alpha$ ``corre'' nel suo campo di definizione.
\vfill
\break
\SectionPacco{Unit\`a Geometrizzate}
Spesso si ha a che fare con grandezze caratteristiche rispetto a cui riferire tutte le altre grandezze ad essa legate (per esempio la velocit\`a della luce permette di scrivere tutte le velocit\`a in maniera adimensionale, limitando i valori possibili fisicamente a quelli dell'intervallo $[0,1]$). Ecco alcune delle grandezze e delle possibili conversioni:
\medskip
{\leftskip 2cm
$$
\eqalign{
1.0&=G=6.670\cdot 10^{-11}\rm cm^3/g\cdot sec^2\cr
1.0&=c=2.997930\cdot 10^{10}\rm cm/sec\cr
1.0&=G/c^2=0.7425\cdot10^{-28}\rm cm/g\cr
1.0&=G/c^4=0.826\cdot10^{-49}\rm cm/erg\cr
1.0&=Gk/c^4=1.140\cdot10^{-65}\rm cm/{}^{\circ}K\cr
1.0&=c^2/\sqrt{G}=3.48\cdot10^{24}\rm cm\cdot gauss\cr
1.0&=k_B=1.38062\cdot10^{23}\rm J/{}^{\circ}K
}
$$
}


