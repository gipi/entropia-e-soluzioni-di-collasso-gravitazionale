\Chapter{Geometrodinamica}{\hfill`` Geometry tells matters how to move, matter tells geometry how to curve.''}{John Arcibald Wheeler}
In questo capitolo verranno presentati gli elementi utili per riuscire a trattare problemi fisici in presenza di gravit\`a, almeno nell'ambito classico della \vip{Relativit\`a Generale}: si partir\`a da un approccio variazionale  per ricavare le equazioni di campo, si introdurr\`a il calcolo tensoriale ed il \vip{Tensore Energia-Impulso} per distribuzioni di materia generiche; si ricaveranno in seguito le soluzioni conosciute nel vuoto ed l'approccio termodinamico alla fisica dei buchi neri sviluppato da \vip{Bekenstein}.
%
%
%
\Section{Principio di equivalenza}
Con il rivoluzionario articolo del 1905 sull'elettrodinamica dei corpi in mo\-vi\-men\-to, \vip{Albert Einstein } eliminava quelle che parevano essere le contraddizioni fra l'elet\-tro\-ma\-gne\-ti\-smo di \vip{Maxwell} e la fisica di \vip{Newton}: il tempo e lo spazio assoluti perdono di significato tramite i postulati della costanza della velocit\`a della luce e dell'equivalenza delle leggi fisiche nei sistemi inerziali e vengono sostituiti da quello che in seguito sar\`a chiamato spazio tempo; l'elettrodinamica ed in particolare le leggi di Maxwell che la descrivono completamente sono riscritte in maniera molto pi\'u elegante grazie a questo nuovo formalismo:
$$
\left\{\eqalign{&\vec{\nabla}\cdot\vec{E}=\rho\cr
&\vec{\nabla}\times\vec{E}=-{\partial\vec{B}\over \partial t}\cr
&\vec{\nabla}\cdot\vec{B}=0\cr
&\vec{\nabla}\times\vec{B}={\partial\vec{E}\over \partial t}+\vec{J}\cr
&\vec{f}=q(\vec{E}+\vec{v}\times\vec{B})
}\right.\quad \hbox{che vengono riscritte come}\quad%
\left\{\eqalign{&\epsilon^{ijkl}\partial_{j}F_{kl}=0\cr
&\partial_{\nu}F^{\mu\nu}=-{4\pi\over c}J^{\mu}\cr
&f^{\mu}=q\Mat{F}{\mu}{\nu}u^{\nu}
}\right.
$$
dove si \`e posto
$$
\left\{\eqalign{
&F_{\mu\nu}=\partial_{\mu}A_{\nu}-\partial_{\nu}A_{\mu}\cr
&E^i=F^{0i}\cr
&B^i=\epsilon^{i\alpha\beta}F_{\alpha\beta}\cr
}\right.
$$
come si pu\`o vedere tramite un solo quadrivettore $A^{\mu}$ si ha una descrizione molto pi\'u elegante di questa teoria.
\par
In tutto il processo di ``ricostruzione'' della fisica (colpita nel frattempo anche dalla neonata meccanica quantistica) non era stata inclusa la teoria della gravitazione, altra teoria di campo (classica) che aveva riscosso molti successi nel descrivere il moto di corpi di grandi dimensioni; infatti il campo gravitazionele $\phi(\vec{r})$ generato da un corpo di densit\`a $\rho$ era determinato dall'equazione di Poisson:
$$
\triangle\phi(\vec{r})=\rho(\vec{r})
$$
Essa ammette una velocit\`a di propagazione infinita ed in pi\'u  non \`e facilmente es\-ten\-di\-bi\-le al formalismo omogeneo in quattro dimensioni.
\par
Questa analisi giungeva in concomitanza del nuovo obiettivo dell'ormai professore Einstein: arrivare ad una teoria che permettesse di scrivere le leggi della fisica in sistemi di riferimento qualsiasi, cio\`e di sganciare la fisica dai sistemi inerziali. Durante i suoi studi arriv\`o al seguente ragionamento:\Foot{Questa traduzione \`e tratta da una raccolta di articoli di fisici Tedeschi del secolo scorso (fra cui Einstein) scaricabile dal sito dell' universit\`a di Roma all'indirizzo {\WWW http://ipparco.roma1.infn.it/pagine/deposito/2002/a.ps.gz}.}
\bigskip
\hbox to \hsize{\hfill\vbox{\hsize=8cm\Italic\noindent [$\dots$] L'introduzione della relativit\`a generale deve con\-dur\-re parimenti ad una teoria della gravitazione; si pu\`o infatti generare un campo gravitazionale con il puro cam\-bia\-men\-to del sistema di coordinate [$\dots$]}\hfill%
}\bigskip
In effetti \`e conosciuto da tutti il fenomeno che si presenta nei sistemi accelerati delle \vip{forze fittizie}: rispetto ad un sistema {\Italic fermo} su cui non agiscono forze, chiamato \vip{sistema inerziale}, un corpo in un  sistema accelerato rispetto ad esso, subisce delle accelerazioni indipendenti dalla costituzione interna dell'oggetto e praticamente in\-distingui\-bi\-li (puntualmente) da forze gravitazionali (si distinguono tra l'altro per il loro comportamento all'infinito: le forze gravitazionali all'infinito si annullano, mentre le accelerazioni ``fittizie'' tendono ad un valore illimitato all'infinito); un esempio pu\`o essere il sistema fisico rappresentato da una giostra in rotazione ad una velocit\`a angolare costante $\omega$: tutti i bambini sanno che se vincolati ad essa, sono soggetti ad una accelerazione costante, detta accelerazione centrifuga, di modulo $a_c={v^2\over r}=\omega^2 r$ mentre i bambini a terra  non subiscono nessuna accelerazione e continuano a mangiare felici il gelato.
Ed \' e proprio qui che entra in gioco il \vip{principio di equivalenza}\ alla base della Relativit\`a Generale: {\sl esiste (sempre e ovunque) un intorno abbastanza piccolo di un evento nello \ST\ in cui tutte le leggi fisiche appaiono indistinguibili da quelle della relativit\`a speciale} oppure espresso in una maniera pi\'u intuitiva, non si pu\`o distinguere localmente (per la precisione puntualmente) fra un campo gravitazionale ed un sistema di riferimento accelerato.\par
Ritornando al sistema della giostra esiste una maniera di collegare il modo in cui tutti i bambini ``vedono'' il mondo: sia $IF(t,r,\theta,z)$ il sistema inerziale di coordinate dei bambini ``fermi'' e  $RF(T,R,\Theta,Z)$ quello dei bambini ``ruotanti'', le loro coordinate sono collegate dalla seguente relazione
$$
\left\{\eqalign{%
&t=T\cr
&r=R\cr
&\theta=\omega t+\Theta\cr
&z=Z
}\right.
$$
L'osservatore inerziale a causa delle contrazioni di Lorentz (indipendenti dall'ac\-ce\-le\-ra\-zio\-ne)\Foot{Orologi non influenzati da accelerazione vengono chiamati \vip{orologi standard}.} misurer\`a intervalli temporali diversi da quelli di RF ed inoltre queste misure differiranno a seconda della distanza dall'asse di rotazione secondo la legge:
$$
dt=\sqrt{1-{\omega^2 r^2\over c^2}}dT
$$
Nel caso vengano  misurati tramite orologi solidali agli osservatori in tutte e due i sistemi,  l'osservatore fermo nel sistema rotante noter\`a anche lui una dipendenza dalla posizione degli intervalli temporali misurati da orologi standard e inputer\`a questo al campo di forze fittizie createsi nel sistema di riferimento a causa del fatto che non \`e inerziale; a questo punto non potendo distinguere puntualmente tra un campo gravitazionale ed un campo fittizio di forze significa che anche il campo gravitazionale modifica lo ``scorrere del tempo''. Ma cosa succede invece agli intervalli spaziali? In IF la metrica \`e quella piatta in coordinate cilindriche
$$
\etaM=-\Tensor{dt}+\Tensor{dr}+r^2\Tensor{d\theta}+\Tensor{dz}
$$
 mentre il cambio di coordinate la porta in una metrica non diagonalizzabile tramite trasformazioni algebriche:
$$
g_{RF}=-\left(1-{\omega^2r^2\over c^2}\right)c^2\tensor{dt}+\tensor{dr}+2r\omega\, dt\otimes d\theta+r^2\tensor{d\theta}+\tensor{dz}
$$
Usando gli appositi strumenti messi a disposizione dalla geometria differenziale si ottiene che in un cerchio nel sistema rotante il rapporto fra la circonferenza e il raggio \`e maggiore di $2\pi$: quindi per il principio di equivalenza anche nei sistemi gravitazionali si ha una modifica delle leggi euclidee della geometria. Einstein a partire da questi ragionamenti  perci\`o si accorse che quello che serviva per eliminare l'arbitrariet\`a delle coordinate, era un formalismo gi\`a sviluppato da tempo da personaggi quali \vip{Gauss}\Foot{Nel 1827 Gauss pubblica ``Disquisitiones generales circa superficies curves'' in cui definisce la curvatura di superfici bidimensionali (curvatura detta gaussiana) da propriet\`a intrinseche delle superfici (Teorema Egregium).}\break \vip{Riemann}\ ed in particolare egli necessitava del nuovo calcolo tensoriale sviluppato dagli italiani \vip{Levi-Civita} e \vip{Ricci-Curbastro}. Nel prossimo capitolo faremo una car\-rel\-la\-ta dei concetti principali di questa branca della matematica.
%
%
%
\input GeomDiff
%
%
%
\Section{Equazioni di Einstein}
Una volta messa a disposizione la geometria differenziale necessaria a formalizzare la teoria della relativit\`a generale come fatto in tutta la sezione precedente, ci ac\-cin\-gia\-mo a ricavare le equazioni di Einstein e le prime soluzioni nel vuoto; prima per\`o un ultimo passaggio sulla geometria differenziale per poter definire il calcolo variazionale.
%
%
%
\SottoSezione{Calcolo Variazionale}
Come gi\`a accenato nelle sezioni precedenti, i sistemi fisici sono trattati come Campi, cio\`e come sistemi ad infiniti gradi di libert\`a aventi certe propriet\`a sotto trasformazioni di coordinate; tutto ci\`o si formalizza a livello geometrico dicendo che le configurazioni di un sistema sono date da sezioni in determinati fibrati a seconda del sistema in esame. Il formalismo moderno della teoria dei campi presuppone che le equazioni che governano un sistema siano date dalla risoluzione di un problema variazionale.\par
\Def{Azione}
Si definisce Azione di una sezione $\rho$ su una regione $D\subseteq\MM$ compatta di dimensione $m$ tramite il seguente integrale
$$
A_D(\rho)=\int_DL\circ j^k\rho 
$$
dove $L\colon J^kC\to\Lambda^\ast_m(\MM)$ \`e detta \vip{Lagrangiana} di ordine $k$.
FineDef
Prendendo un flusso qualsiasi di trasformazioni $\varphi_s$ verticali che si annullino al bordo di $D$ insieme alle proprie derivate fino all'ordine $(n-1)$, si pu\`o definire la variazione dell'azione e richiedere che per $\rho$ tali che la variazione sia nulla siano soluzioni classiche (dette anche on-shell)
$$
\delta_\xi A_D(\rho)={d\over ds}\Big|_{s=0}A_D(\rho_s)
$$
questo porta alla definizione delle equazioni di \vip{Eulero~-Lagrange} che per una teoria del primo ordine restituiscono le ben note equazioni:
$$
{\partial L\over \partial y^\mu}-d_\nu{\partial L\over\partial \Mat{y}{\mu}{\nu}}=0\Eqno
$$
\COUNT{Eulero}
\Def{Simmetrie della lagrangiana}
Dato un flusso di trasformazioni sul fibrato delle configurazioni
$$
\left\{\eqalign{%
&x^{\prime\mu}=\varphi^{\mu}_s(x)\cr
&y^{\prime\alpha}=\Phi_s^\alpha(x,y)
}\right.
$$
che induce un campo vettoriale $\Xi=\xi^\mu\PT{\mu}+\xi^\alpha\PT{\alpha}$ proiettabile su un campo $\xi\in\hbox{\Gotico X}(\MM)$ si definisce simmetria della lagrangiana $L$ di ordine $k$ se e solo se 
$$
J_s{\cal L}\circ j^k\Phi_s={\cal L}
$$
dove ${\cal L}$ sono le componenti della lagrangiana e $J_s$ \`e il determinante dello jacobiano generato da $\varphi_s$.
FineDef
La vera importanza del calcolo variazionale \`e la possibilit\`a di ricavare tutte le quantit\`a conservate a partire dalle simmetrie del sistema in esame tramite il \vip{teorema di N\"other}. Infatti \`e possibile scrivere la variazione dell'azione come somma di due cosidetti morfismi variazionali: {\Gruppi E}$(L)$ detto \vip{morfismo di Eulero-Lagrange}\ e {\Gruppi F}$(L)$ chiamato \vip{morfismo di Poincar\'e-Cartan}
$$
\delta_\xi A_D(\rho)=<\hbox{\Gruppi E}(L)|\xi>+\hbox{Div}<\hbox{\Gruppi F}(L)|j^{k-1}\xi>\Eqno
$$
\COUNT{FVF}
valevole $\forall\xi$ e chiamata \vip{formula della variazione prima}; nel caso il campo $\xi$ sia una simmetria allora vale  \vip{l'identit\`a di covarianza}\ 
$$
\delta_{\Lie{y}{\Xi}} L=\hbox{Div}(i_\xi L)
$$
che permette di riscrivere grazie alla \FVF\ i risultati precedenti in una particolare forma
$$
\hbox{Div}{\cal E}(L,\Xi)={\cal W}(L,\Xi)\Eqno
$$
\COUNT{EW}
dove
$$
\cases{%
{\cal E}(L,\Xi)=<\hbox{\Gruppi F}(L)|j^{k-1}\Lie{y}{\Xi}>-i_\xi L\quad&$\triangleleft$ \hbox{Corrente di N\"other}\cr
{\cal W}(L,\Xi)=-<\hbox{\Gruppi E}(L)|j\Lie{y}{\Xi}>\quad&$\triangleleft$ \hbox{Forma Lavoro}
}
$$
\par
A questo punto \`e utile soffermarsi ad analizzare quello che \`e stato trovato: a causa della sua definizione ${\cal W}(L,\Xi)$ \`e nullo applicato su una soluzione e questo porta ad una legge di conservazione {\sl on shell} (cio\`e lungo sezioni critiche) per ${\cal E}(L,\Xi)$ che per questo viene chiamata ``corrente debole''; a differenza della meccanica classica dove le quantit\`a conservate erano funzioni scalari (cio\`e $0$-forme), qui abbiamo a che fare con $(n-1)$-forme differenziali e quindi per ottenere lo stesso risultato si ha bisogno di integrarle rispetto ad ipersuperfici dello \ST, per cui sarebbe utile trovare forme chiuse in maniera che gli integrali risultino alla fine di tutto indipendenti dall'integrazione.\par
Nel caso trattassimo cosidette {teorie naturali} avremo la possibilit\`a di decomporre ulteriormente sia la corrente di N\"other che la forma lavoro ottenendo i seguenti importanti risultati
$$
\eqalign{%
&{\cal E}(L,\xi)=\tilde{\cal E}(L,\xi)+\hbox{Div}{\cal U}(L,\xi)\cr
&{\cal W}(L,\xi)=<{\cal B}(L)|\xi>+\hbox{Div}\tilde{\cal E}(L,\xi)
}
$$
dove $\tilde{\cal E}(L,\xi)$ \`e chiamata \vip{corrente ridotta}\ ed \`e nulla on shell, ${\cal U}(L,\xi)$ \`e chiamato \vip{superpotenziale}; confrontando con la \EW\ si ottiene la particolare regola valevole anche {\sl off shell}, cio\`e lungo una qualunque configurazione e quindi di natura geometrica
$$
{\cal B}(L)=0 \quad\triangleleft\hbox{Identit\`a di Bianchi generalizzate}
$$
\Def{Quantit\`a conservate}
Consideriamo una teoria naturale con una lagrangiana $L$; sia $\Omega\subset\MM$ una regione $(m-2)$-dimensionale, $\xi$ un generatore di simmetria della lagrangiana, ${\cal U}(L,\xi)$ il superpotenziale relativo ed infine $\rho$ una sezione soluzione dell'equazioni di campo.
$$
Q_\Omega(\xi,\rho)=\int_\Omega{\cal U}(L,\xi,\rho)=\int_\Omega(j^{2k-2}\varphi)^\ast{\cal U}(L,\xi)
$$
\`e detta \vip{quantit\`a  covariantemente conservata}.
FineDef
Bisogna aggiungere che la parola {\sl covariantemente} indica il fatto che l'integrale di definizione \`e invariante   rispetto ai cambiamenti di coordinate e quindi non dipende in nessun caso dal sistema di coordinate usate per effetture il calcolo effettivo, inoltre \`e  un invariante omologico, cio\`e non dipende dalla deformazione della regione d'integrazione. Infatti considerando due superfici incapsulate $\Omega_1$ e $\Omega_2$  tali che a livello omologico $\Omega_2-\Omega_1$ sia un bordo $(m-1)$ dimensionale $\partial D$ (che contengano entrambi le stesse singolarit\`a ma tali che non vi siano singolarit\`a tra loro) allora si ha che 
$$
\eqalign{%
Q_{\Omega_2}-Q_{\Omega_1}&=\int_{\Omega_2-\Omega_1}{\cal U}(L,\xi,\rho)\cr
&=\int_{\partial D}{\cal U}(L,\xi,\rho)\cr
&=\int_{D}d{\cal U}(L,\xi,\rho)\cr
&=\int_D{\cal E}(L,\Xi,\rho)
}
$$
siccome ${\cal E}(L,\Xi,\rho)$ soddisfa ad una equazione di continuit\`a la quantit\`a $Q_\Omega$ \`e una {\sl quantit\`a conservata} e le singolarit\`a sono viste come sorgenti.
\Figura{noether}{6cm}{6cm}{Superfici incapsulate intorno ad una singolarit\`a.}
\Example{Yang-Mills Abeliano}
Prendiamo la lagrangiana per una teoria di Yang-Mills Abeliana (cio\`e scritta su {\Gruppi U}$(1)$) su una variet\`a quadridimensionale $(\MM,\etaM)$
$$
L_{YM}=-{1\over4}\etaM^{\mu\rho}\etaM^{\nu\sigma}F_{\mu\nu}F_{\rho\sigma}\,d\mu
$$
si ricavano il morfismo di Eulero-Lagrange e la corrente di N\"other
$$
\eqalign{
&\hbox{\Gruppi E}(L_{YM})=\nabla_\nu\left(\sqrt{|g|} F^{\mu\nu}\right)\cr
&{\cal U}(L_{YM},\xi_{(V)})=-{1\over2}F^{\mu\nu}\xi_{(V)}d\mu_{\mu\nu}\cr
&\tilde{\cal E}(L_{YM},\xi_{(V)})=-\DC{\mu}\left(\sqrt{|g|}F^{\mu\nu}\right)\xi_{(V)}
}
$$
dove $\xi_{(V)}$ \`e il generatore infinitesimo della trasformazine di gauge che manda il campo $A_\mu$ soluzione delle equazioni di campo in un altra soluzione generata dal tensore $F_{\mu\nu}=\partial_{\mu}A_{\nu}-\partial_{\nu}A_{\mu}$ scomponibile in due tipi di vettori tridimensionali tramite la seguente identificazione
$$
\eqalign{
&E^a=F^{0a}\cr
&B^a=\epsilon^{abc}F_{bc}
}
$$
mentre $d\mu$ e $d\mu_{\mu\nu}$ sono state definite tramite le \FVolume\ e \FVolumeTwo. Utilizzando il su\-per\-po\-ten\-zia\-le sopra definito con questa identificazione si ritrova una ben nota formula dell'elet\-tro\-di\-na\-mi\-ca integrando il superpotenziale rispetto ad una superficie spaziale $S$ contenente una carica elettrica puntiforme che genera il campo elettrico $\vec E$
$$
Q=\int_S{\vec E}\cdot{\vec n}\,d\Sigma
$$
La teoria del superpotenziale estende a tutte le correnti di N\"other questa visione {\Italic \'a la Gauss}.
FineEx
%
%
%
\SottoSezione{Equazioni di campo della Relativit\`a Generale}
Applicando quanto detto nella sezione precedente partiamo a calcolare le equazioni che governano l'evo\-lu\-zio\-ne della geometria quadridimensionale.\par
Fisicamente sappiamo che la teoria deve essere impostata su una variet\`a qua\-dri\-di\-men\-sio\-na\-le su cui \`e definita una metrica di segnatura $(-,+,+,+)$ che per\`o nel caso in questione si eleva a campo fisico ed \`e pertanto a priori incognito; quindi la lagrangiana di cui abbiamo bisogno deve contenere la metrica oltre che le sue derivate sotto forma di uno scalare; la possibilit\`a pi\'u semplice risulta essere
$$
L_H={1\over \kappa}\sqrt{-g}R\,d\mu
$$
\COUNT{LagrangianaHilbert}
in quanto $R$ (scalare di Ricci) \`e l'unico tensore scalare che contiene  derivate della metrica; in realt\`a entrano in gioco anche derivate seconde della metrica che ge\-ne\-re\-reb\-be\-ro a priori equazioni contenenti derivate rispetto al tempo del quarto ordine della metrica. Per\`o le derivate seconde entrano linearmente in $R$ e possono essere ricombinate in una quadridivergenza facendo risultare la grandezza
$$
G={1\over \kappa}g^{\alpha\beta}(\GC{\mu}{\alpha}{\sigma}\GC{\sigma}{\beta}{\mu}-\GC{\sigma}{\alpha}{\beta}\GC{\mu}{\sigma}{\mu})\Eqno
$$\COUNT{NonCov}
generatrice delle suddette equazioni (bisogna notare che questa grandezza non ha carattere tensoriale, ma su questo ci torneremo in seguito).\par
Aggiungiamo a questa una lagrangiana $L_M$ relativa alla materia  e applichiamo Eulero-Lagrange, si ottengono come equazioni le seguenti
$$
R_{\mu\nu}-{1\over 2}g_{\mu\nu}R=\kappa T_{\mu\nu}\Eqno
$$
\COUNT{EqEi}
dove il termine di sinistra viene chiamato \vip{tensore di Einstein} (in seguito sar\`a indicato con $G_{\mu\nu}$) ed il termine di destra che opera da sorgente della curvatura dello spazio tempo viene chiamato \vip{tensore energia-impulso} o tensore di stress di Hilbert e si ricava tramite l'equazione
$$
T_{\mu\nu}={\delta L_M\over \delta g^{\mu\nu}}
$$
dove con $\delta/\delta g^{\mu\nu}$ si indica la derivata variazionale rispetto al campo $g^{\mu\nu}$, mentre  $\kappa$ \`e una costante che ricaveremo in seguito. Questo tensore simmetrico per costruzione ha una importanza essenziale in quanto le sue componenti rappresentano la densit\`a di quadrimpulso presente in una certa regione di \ST. Preso infatto un quadrivettore $u$ rappresentante un osservatore e una terna di vettori di tipo spazio $\{e_\alpha\}$ perpendicolari rispetto ad esso 
$$\eqalign{
&T(u,\cdot)=-\left(\vcenter{\hsize 7cm \Caption \noindent Densit\`a di quadrimomento per unit\`a di volume tridimensionele ${d{\vec p}\over dV}$ come misurata nel sistema inerziale di un osservatore rap\-pre\-sen\-ta\-to dal vettore $u$.}\right)\cr
&T(e_\alpha,\cdot)=\left(\vcenter{\hsize 5cm \Caption \noindent Forza per unit\`a di superficie identificata da $e_\alpha$ misurata da un osservatore rappresentato da~$u$}\right)
}
$$ 
 Importante \`e la particolarit\`a che il tensore di Einstein \`e a divergenza nulla a causa dell'identit\`a di Bianchi e questo si trasporta direttamente su $T_{\mu\nu}$, ma queste non sono altro che le leggi di conservazione dell'energia e dell'impulso e quindi contengono in s\'e le equazioni del moto del sistema descritto da $L_M$: \vip{quindi le equazioni del campo gravitazionale contengono in s\'e anche le leggi di  conservazione relative alla materia stessa che genera questo campo}; la distribuzione e il moto della materia che curva lo spazio non possono essere dati arbitrariamente ma debbono essere determinati contemporaneamente al campo stesso creato da questa materia.\par
In particolare se prendiamo un campo vettoriale ${\bf u}$ su $\MM$ rappresentante una famiglia di osservatori in ``caduta libera'' e non interagenti, il loro tensore energia impulso sar\`a rappresentato da un tensore del tipo
$$
T^{\alpha\beta}=\epsilon u^\alpha u^\beta\Eqno
$$
\COUNT{TEIos}
dove $\epsilon$ \`e la densit\`a di energia misurata localmente dagli osservatori stessi; richiedendo la condizione di quadridivergenza nulla, contraendo con $u$ l'indice rimasto libero si ottiene 
$$
\eqalign{
u_\beta\DC{\alpha}T^{\alpha\beta}&=u_\beta\DC{\alpha}(\epsilon u^\alpha u^\beta)\cr
&=u_\beta u^\beta\DC{\alpha}(\epsilon u^\alpha)+\epsilon u_\beta u^\alpha\DC{\alpha}u^\beta\cr
&=0
}
$$
ma dalla condizione di fisicit\`a delle curve di moto degli osservatori\Foot{Sappiamo che in un sistema inerziale per il principio di equivalenza si deve avere $u_\alpha u^\alpha=-1$ e siccome \`e una quantit\`a scalare deve conservare il suo valore (costante) in ogni altro sistema di riferimento.} cio\`e imponendo $u_\beta\DC{\alpha}u^\beta=0$ si ritrova $\DC{\alpha}(\epsilon u^\alpha)=0$ e sostituendo in $\DC{\alpha}T^{\alpha\beta}=0$ si ottiene finalmente
$$
u^\alpha\DC{\alpha}u^\beta=0
$$
che \`e l'equazione delle \vip{geodetiche} ottenibili come problema variazionale dalla la\-gran\-gia\-na
$$
L_\gamma=\sqrt{g_{\alpha\beta}u^\alpha u^\beta}\,ds
$$
\Example{Campo Elettromagnetico}
Nel caso del campo elettromagnetico si ha una lagrangiana data da 
$$
L_{EM}=-{1\over4}\sqrt{|g|}\,g^{\mu\rho}g^{\nu\sigma}F_{\mu\nu}F_{\rho\sigma}\,d\mu
$$
da cui con le regole appena spiegate si ricava
$$
T^{(EM)}_{\mu\nu}=F_{\mu\alpha}\Mat{F}{\alpha}{\nu}-{1\over4}F_{\alpha\beta}F^{\alpha\beta}g_{\mu\nu}
$$
FineEx
Per quanto riguarda le quantit\`a conservate il fatto che il tensore energia impulso sia a quadridivergenza nulla non implica nessuna quantit\`a fisica conservata a meno che sia assente campo gravitazionale in quanto 
$$
\DC{\mu}T^{\mu\nu}={1\over\sqrt{-g}}\partial_\mu (\sqrt{-g}T^{\mu\nu})-{1\over2}\partial^\nu g_{\alpha\beta} T^{\alpha\beta}=0
$$
Per ottenere delle quantit\`a conservate si  utilizza il formalismo della sezione precedente e si giunge a definire il \vip{superpotenziale di Komar}
$$
{\cal U}(L_H,\xi)={1\over2k}\sqrt{-g}\nabla^{\alpha}\xi^\beta d\mu_{\alpha\beta}\Eqno
$$
\COUNT{Komar}
quantit\`a geometrica su cui torneremo in seguito.\par
Le equazioni \EqEi\ sono 10 equazioni differenziali alle derivate parziali non li\-nea\-ri e per questo sono molto difficili da risolvere (ad oggi si conoscono appena una decina di soluzioni per il problema ad un corpo, mentre per il problema dei due corpi si hanno solo simulazioni numeriche) ed in particolare, a differenza degli altri campi conosciuti quali per esempio il campo elettro-magnetico, non vale il principio di sovrapposizione; per quanto riguarda il problema delle condizioni iniziali, studiando in dettaglio il problema \Biblio{Land02}{L.\ D.\ Landau E.\ M.\ Lif\v sits}{Teoria dei campi}{Secondo volume della collana Fisica Teorica-Editori Riuniti}\ si giunge alla conclusione che si possono dare arbitrariamente i valori delle componenti spaziali della metrica per poi ricavare i valori di $g_{0\mu}$ ammissibili. Questo modo di operare viene chiamato \vip{decomposizione ADM}: viene scelto un campo vettoriale timelike  $\zeta$ sopra $(\MM,g)$ ed il flusso che lo genera (chiamiamolo $\varphi_t$) rappresenter\`a l'evoluzione quadridimensionale di un punti nel sistema di riferimento dell'osservatore; a questo punto l'osservatore sceglie una ipersuperficie di tipo spazio $\Sigma$ che rappresenta il suo mondo tridimensionale che si evolver\`a di conseguenza tramite $\varphi_t(\Sigma)$.
\Figura{ADM}{3.5cm}{5cm}{Decomposizione ADM.}
$\Sigma$ viene chiamata \vip{superficie di Cauchy}\ e rappresenta in pratica le condizioni iniziali del sistema, affinch\'e sia possibile decomporre la variet\`a $\MM$ in un prodotto del tipo $\RN{}\times\Sigma$ si ha bisogno che $\MM$ sia \vip{globalmente iperbolico}.\Foot{L'importanza di questa condizione risiede nel fatto che in variet\`a con queste caratteristiche, il problema delle condizioni iniziali \`e ben posto.}
%
%
%
\par
Andiamo ad analizzare alcune soluzioni tipiche della relativit\`a generale:
%
%
%
\SottoSezione{Equazioni linearizzate}
Come necessario ad ogni teoria fisica nuova, deve riprodurre in un determinato limite la teoria precedente, in questo caso la teoria della gravitazione universale di Newton\Foot{Nello stesso modo  la teoria di Newton incorporava in se la teoria degli epicicli greci.}.\par
Consideriamo una metrica che si discosta leggermente da quella di \Mink\ 
$$
g_{\mu\nu}=\etaM_{\mu\nu}+h_{\mu\nu}\quad\hbox{con}\quad|h_{\mu\nu}|\ll1
$$
e calcoliamo le quantit\`a geometriche che ne derivano attenendoci ad un'approsimazione del primo ordine
\bigskip
\Item{}\ $\Gamma_{\alpha\mu\nu}={1\over2}\left(\partial_\nu h_{\alpha\mu}+\partial_\mu h_{\alpha\nu}-\partial_\alpha h_{\mu\nu}\right)$
\Item{}\ $R_{\alpha\mu\beta\nu}={1\over2}\left(\partial_{\mu\beta}h_{\nu\alpha}+\partial_{\alpha\nu}h_{\mu\beta}-\partial_{\alpha\beta}h_{\mu\nu}-\partial_{\mu\nu}h_{\alpha\beta}\right)$
\Item{}\ $R_{\mu\nu}={1\over2}\left[\partial_{\alpha\mu}\Mat{h}{\alpha}{\nu}+\partial_{\alpha\nu}\Mat{h}{\alpha}{\mu}-\partial_{\mu\nu}h-\Box h_{\mu\nu}\right]$
\Item{}\ $R=\partial_{\mu\nu}h^{\mu\nu}-\Box h$
\bigskip\noindent
Da queste quantit\`a si pu\`o ricavare il corrispondente tensore di Einstein 
\bigskip
\Item{}\ $G_{\mu\nu}={1\over2}\left[\partial_{\alpha\mu}\Mat{h}{\alpha}{\nu}+\partial_{\alpha\nu}\Mat{h}{\alpha}{\mu}-\partial_{\mu\nu}h-\Box h_{\mu\nu}-\eta_{\mu\nu}\left(\partial_{\alpha\beta}h^{\alpha\beta}-\Box h\right)\right]$
\bigskip
Tenedo conto che una trasformazione di coordinate induce una variazione nelle componenti di $h_{\mu\nu}$ tramite
$$
h_{\mu\nu}^\prime=h_{\mu\nu}+\partial_{(\mu}\xi_{\nu)}
$$
senza cambiare la situazione fisica, possiamo introdurre il nuovo tensore
$$
\overline h_{\mu\nu}=h_{\mu\nu}-{1\over2}\eta_{\mu\nu}h
$$
e imporre una trasformazione tale che
$$
\partial_\mu \Mat{\overline h}{\mu}{\alpha}=0
$$
A questo punto le equazioni di Einstein al primo ordine risultano essere
$$
\Box \overline h_{\mu\nu}=-2\kappa T_{\mu\nu}\Eqno
$$\COUNT{EqLin}
Per ottenere la legge di gravitazione universale di Newton, consideriamo una metrica statica e supponiamo che i corpi si muovano ad una velocit\`a piccola rispetto a quella della luce per cui possiamo porre $d\tau\sim dt$; l'unico termine che rimane nell'equazione delle geodetiche risulta essere
$$
{1\over c^2}{d^2x^\mu\over dt^2}+\GC{\mu}{0}{0}
$$
Per quanto riguarda il tensore energia-impulso, l'unico termine importante risulta essere la densit\`a di energia
$$
T\sim \epsilon
$$
Tenedo conto di questo nella \EqLin\ si ottiene
$$
\triangle\phi=\kappa\epsilon
$$
e si ricava il valore di $\kappa$ confrontando con la legge della gravitazione newtoniana per ottenere
$$
\kappa={8\pi G\over c^4}\Eqno
$$\COUNT{CostGrav}
Ovviamente la soluzione generale \`e data dalla 
$$
\overline h_{\mu\nu}(t,\vec x)={\kappa\over 2\pi}\int d^3\vec x^\prime {T_{\mu\nu}(t-|\vec x-\vec x^\prime|/c,\vec x^\prime)\over |\vec x-\vec x^\prime|}
$$
che pu\`o essere risolta considerando
\bigskip
\Item{}\ $|T_{00}|\gg|T_{ab}|$
\Item{}\ $|T_{0a}|\gg|T_{ab}|$
\bigskip
e ottenendo
$$
g=-\left(1+{2\phi\over c^2}\right)c^2\Tensor{dt}-{4\over c}A_a dx^a\otimes dt+\left(1-{2\phi\over c^2}\right)\delta_{ab}dx^a\otimes dx^b\Eqno
$$\COUNT{MetrLin}
come metrica generica attorno ad un corpo debolmente rotante.
%
\SottoSezione{\Sch}
Prima soluzione esatta trovata da \Sch nel 1917, un anno dopo la pub\-bli\-ca\-zio\-ne dell'articolo di Einstein; si tratta di una soluzione statica nel vuoto e a simmetria sferica nelle coordinate $(t,r,\phi,\theta)$ che esplicitata diventa:
$$
g_{\lower 1ex\hbox{$\scriptstyle Sch$}}=-\Big(1-{2MG\over c^2r}\Big)\;\tensor{dt}+{1\over {1-{2MG\over c^2 r}}}\,\tensor{dr}+r^2(\tensor{d\theta}+\sin^2\!\theta\,\tensor{d\phi})
$$   
Si nota subito che per ${M\over r}\to 0$ (cio\`e per piccole masse o grandi distanze da queste) si ritrova $\etaM$ e quindi le coordinate diventano rispettivamente il tempo proprio, la distanza radiale e le coordinate angolari sferiche\Foot{In pratica le coordinate sono quello che sembrano. Per un approfondimento sul significato fisico di queste coordinate rimando al [GRAV] ed in particolare al capitolo $23$.} ``misurate'' a distanze molto grandi dal corpo che genera questo campo (in seguito si formalizzer\`a in maniera pi\'u adeguata questo ``infinito'' spaziale).\par
L'importanza nello studio di questa metrica risiede nel fatto che \`e la pi\'u semplice delle metriche ed \`e l'unica a descrivere  un corpo massivo sferico del quale conosciamo bene il campo gravitazionale nel caso Newtoniano; servir\`a il suo studio ad introdurci nelle ``novit\`a'' della teoria Einsteiniana; chiamiamo questo limite, \vip{limite Newtoniano}.\par
Si nota che questa metrica si comporta in maniera singolare per\Foot{Ovviamente in unit\`a geometrizzate (vedi notazioni).} $r=2M$ dove $g_{tt}\to 0$ e $g_{rr}\to \infty$; dal punto di vista fisico la prima condizione significa che gli intervalli temporali per un osservatore stazionario (e dunque di coordinate $r$,$\theta$ e $\phi$ costanti) alla coordinate radiale corrispondente a $2M$ ``visti'' all'infinito si dilatano facendo apparire fermi gli oggetti che si avvicinino a questa superficie: d'ora in poi questo tipo di superfici saranno chiamate superfici di \vip{red-shift infinito}. La condizione per $g_{rr}$ pu\`o essere compresa meglio analizzando la struttura locale dei coni luce: per i fotoni aventi puro moto radiale la condizione $g_{\mu\nu}u^{\mu}u^{\nu}=0$ si traduce in
$$
{dr\over dt}=\pm\big(1-{2M\over r}\big)
$$
da cui si pu\`o osservare che i coni luce collassano sempre per $r=2M$; siccome i coni luce individuano la struttura causale essendo la zona al loro interno l'unica accessibile alle particelle fisiche visto che queste devono avere sempre una velocit\`a inferiore a quella della luce, all'interno di questa superficie (che d'ora in poi sar\`a chiamato \vip{orizzonte degli eventi}) nessuna particella potr\`a mai uscire.
\Figura{OE}{6cm}{9cm}{Rappresentazione del collasso del cono luce all'attraversamento dell'\Oe. Una vol\-ta pas\-sa\-to l'orizzonte degli eventi avviene uno scambio tra la coordinata temporale e quella radiale che costringe ad andare verso raggi sempre minori (in un certo senso cambia anche il senso del tempo).}
Questa si riveler\`a essere la caratteristica pi\'u saliente di questi oggetti che d'ora in poi verranno chiamati \vip{buchi neri}; per entrare pi\'u in dettaglio sul comportamento delle cosidette particelle di prova soggette all'influenza di corpi fortemente massivi consideriamo  i seguenti vettori di Killing di questa metrica: 
$$
\left\{\eqalign{%
&\vec{\xi} _{(t)}=\Vec{e}_t\cr
&\vec{\xi}_{(lz)}=\Vec{e}_{\phi}\cr
}\right.
$$
da cui si pu\`o semplificare l'espressione del quadrimomento grazie a queste due quantit\`a conservate
$$
\left\{\eqalign{
&p\cdot\partial_t\to\epsilon=m\left(1-{2M\over r}\right){dt\over ds}\cr
&p\cdot\partial_\phi\to L=mr^2\sin^2\!\theta{d\phi\over ds}
}\right.
$$
rispettivamente l'energia e il momento angolare misurati da un osservatore all'infinito spaziale; inoltre questa simmetria fa s\'\i\ che il moto avvenga totalmente in un piano, rendendo praticamente ininfluente la coordinata $\theta$: si pu\`o sempre scegliere le coor\-di\-na\-te in maniera tale che\Foot{Un approccio pi\'u intuitivo consiste nel pensare di tracciare il piano contenente il vettore impulso (tridimensionale) passante per il centro del corpo ``attrattore''; questa sar\`a l'unica direzione preferenziale del sistema e non potr\`a variare: fissiamo questo piano essere il piano $\theta={\pi\over 2}$.} ${d\theta\over ds}=0$. \par
\Figura{redshift}{9cm}{7cm}{La conservazione dell'energia misurata all'in\-fi\-ni\-to fa s\'\i\ che sia presente il fenomeno del red-shift, cio\`e man mano che il corpo si avvicina al corpo attrattore la sua energia (misurata localmente) diminuisce.}
Grazie ad esso si pu\`o scrivere il quadrimpulso come:
$$
p^\mu=\Big({\epsilon\over {1-{2M\over r}}},m{dr\over ds},0,{L\over r^2\sin^2\theta}\Big)
$$
\par
Ora entra in gioco la condizione di fisicit\`a delle curve, cio\`e:
$$
\eqalign{%
g_{\mu\nu}p^\mu p^\nu&=-m^2\cr
&={\epsilon^2\over 1-{2M\over r}}+{m^2\over\Rs}\big({dr\over ds}\big)^2+{L^2\over r^2\sin^2\theta}
}
$$
dalla quale isolando il termine $({dr\over ds})^2$ (assimilabile classicamente all'energia cinetica) e scrivendo ripsetto all'energia e al momento angolare per unit\`a di massa (indicate con $\hat{\epsilon}$ e $\hat{L}$) \`e possibile scrivere un potenziale``efficace'' a cui \`e sottoposto il corpo in questione:
$$
\Big({dr\over ds}\Big)^2=\hat{\epsilon}^2-V^2(r)\qquad \hbox{ dove }\qquad V^2(r)=\Big(1-{2M\over r}\Big)\Big(1+{\hat{L}^2\over r^2}\Big)
$$
\Figura{Potenziale}{200pt}{200pt}{Potenziale gravitazionale relativistico di \Sch. Le curve sono identificate nella legenda dal rapporto fra momento angolare e massa.}
Analizzando analiticamente questo potenziale ci si accorge in particolare che i minimi e i massimi (e la loro stabilit\`a) dipendono solamente dal rapporto ${L\over 2M}$, e sono individuati tramite la relazione:
$$
r_{\pm}={L^2\over 2M}\Big(1\pm\sqrt{1-12{M^2\over L^2}}\Big)
$$
\par Volendo ottener esplicitamente la equazione della traiettoria riscriviamo rispetto alla variabile $u=u(\phi)={1\over r(\phi)}$ ed otteniamo
$$
{d^2u\over d\phi^2}+u={M\over L^2}+3Mu^2
$$
con il termine $u^2$ che d\`a la correzione relativistica all'equazione Newtoniana; infatti tralasciando l'ultimo termine si ha come soluzione:
$$
u_0(\phi)={M\over L^2}\big(1+(\alpha-1)\cos\phi\big)\qquad \hbox{dove }\,\alpha={L^2\over MR}\hbox{ con $R$ semiasse maggiore}
$$
a cui corrisponde una traiettoria data dall'equazione $r(\phi)={L^2\over M}{1\over 1+(\alpha -1)\cos\phi}$, cio\`e una conica avente l'eccentricit\`a individuata dal termine davanti al coseno: $e=\alpha -1={L^2\over MR}-1$ ed in particolare si ha:
$$
\left\{\eqalign{%
& e< 1\hbox{ ellisse}\cr
& e=1\hbox{ iperbole}\cr
& e>1\hbox{ parabola}
}\right.
$$
\par Includendo invece il termine relativistico non esistono soluzioni generali, possiamo sperare che la traiettoria non cambi qualitativamente oltre al fatto di ammettere un precessione dell'orbita dovuta al fatto che \`e andato perso il potenziale $\sim {1\over r}$ unico oltre a quello $\sim r^2$ che ammette orbite chiuse. Tentiamo una soluzione del tipo $u(\phi)={M\over L^2}\big(1+e\cos\omega\phi\big)$ con $\omega$ termine molto vicino ad uno \Biblio{EGTO}{O. Gron S. Hervik}{Einstein's General Theory of Relativity}{Download dal sito http://www.fys.uio.no/$\sim$ sgbjorh/GRbook.html}:
$$
\delta\omega={3(GM)^{3\over 2}\over c^2(1-e^2)a^{5\over2}}
$$
\Figura{traiettoria2}{200pt}{200pt}{Esempio di geodetiche timelike con la loro traiettoria a forma di rosetta.}
Rifacendo tutto il discorso per le particelle a massa nulla quali per esempio fotoni e gravitoni, utilizzando la condizione on-shell:
$$
g_{\mu\nu}u^{\mu}u^{\nu}=0
$$
si ottiene\Foot{Notare come il potenziale dipenda da $b$ che non \`e altro che il rapporto fra l'energia ed il momento angolare.}una nuova equazione radiale \Biblio{GRAV}{C. W. Misner K.S. Thorne J. A. Wheeler}{Gravitation}{W. H. Freeman and Company}:
$$
\big({dr\over ds}\big)^2={1\over b^2}-{1\over r^2}\big(1-{2M\over r}\big)\quad \hbox{dove $b$ \`e il parametro di impatto}
$$
con un potenziale efficace avente un massimo ad un raggio $r=3M$ mentre l'equazione della traiettoria risulta:
$$
{d^2u\over d\phi^2}+u=3Mu^2
$$
Risolvendo con la stessa procedura precedente si ottiene
$$
u={1\over b}\left[\cos\phi+{M\over b}\left(1+\sin^2\phi\right)\right]
$$
dal quale si pu\`o ricavare come valore di deflessione $\delta\phi$ di una particella passante ad una distanza $b$ da un corpo di massa $M$
$$
\delta \phi={4M\over b }
$$
\Figura{TraietFotone}{12cm}{4cm}{Traettoria incurvata di una particella a massa nulla nei pressi di un corpo massivo.}
\par
Abbiamo detto che una volta superata dall'esterno la superficie identificata da $r={2GM\over c^2}$ non se ne pu\`o pi\'u uscire, ma allora sorge da parte dell'osservatore esterno il problema di poter descrivere queste parti di \ST\  visto che un corpo eventualmente nelle vicinanze di questa superficie ed in grado di emettere segnali luminosi per ``comunicare'' con l'esterno avr\`a in realt\`a dei problemi nel farlo: infatti la relazione per corpi a massa nulla fra tempo e coordinata radiale (nel caso di puro moto radiale) \`e data da:
$$
dt={dr\over 1-{2M\over r}}\qquad\hbox{cio\`e}\qquad\Delta t=r+2M\log\|r/2M-1\|
$$
che come si pu\`o vedere chiaramente per la formula da un infinito all'approcciarsi dell'\Oe; lo stesso discorso si pu\`o fare per particelle aventi un momento angolare e/o massive: quindi le particelle che attraversano questa superficie si trovano ad un ``infinito temporale'' rispetto all'osservatore esterno ed in pratica a livello ottico, per il risultato appena discusso, tutte le cose cadute nel buco nero rimangono congelate sul suo \Oe.\par
Ma allora dove sta il trucco? Come si possono descrivere descrivere le porzioni di \ST\ all'interno di esso? Sempre dalla relazione precedente si pu\`o definire un nuova coordinata: $r^{\ast}=r+2M\log\|r-2M\|$ detta \vip{coordinata della tartaruga}\ introdotta da Wheeler nel 1957 con la quale \`e possibile definire due famiglie di geodetiche nulle:
$$
u=t-r^\ast\qquad v=t+r^\ast
$$
ed una nuova espressione della metrica in queste coordinate:
$$
g_{KS}=-\Big(1-{2M\over r(u,v)}\Big)\,du\,dv+r^2(u,v)(\tensor{d\theta}+\sin^2\!\theta\,\tensor{d\phi})
$$
che rende evidente il fatto che il valore $r=2M$ \`e solo una singolarit\`a delle coordinate.
%
%
%
\SottoSezione{Kerr}
Soluzione trovata per primo da \Biblio{Kerr63}{R.P.Kerr}{Gravitation Field of spinning mass as an examples of algebraically special metrics}{Phys.Rev.Letters 11,237 (1963)} e poi scritta nella forma:
$$
\eqalign{%
g_{KN}=-\left(1-{2Mr\over\rho^2}\right)&\tensor{dt}+{\rho^2\over\Delta}\tensor{dr}+\rho^2\tensor{d\theta}\cr
&+\left(r^2+a^2+{2Mra^2\over\rho^2}\sin^2\theta\right)\sin^2\!\theta\;\tensor{d\phi}-{4Mra\over\rho^2}\sin^2\theta\; d\phi\otimes dt
}
$$
\line{\hfill $\Delta=r^2-2Mr+a^2$\hskip 1cm $\rho^2=r^2+a^2\cos^2\theta$\hfill}
da Boyer e Lindquist\Foot{La metrica trovata da Kerr era in una forma pi\'u complicata e la verifica come soluzione risultava lunga e laboriosa; \Biblio{Cart71}{B.Carter}{Axisymmetric black hole has only two degree of freedom}{Phys.Rev.Lett.26,331(1971)} ha dimostrato che \`e l'unica soluzione a simmetria assiale rappresentante un buco nero (o in generale un corpo massivo) rotante.} e pu\`o considerarsi la metrica nell'intorno di un corpo rotante\Foot{Per convincersi di questo fatto basta confrontare il termine misto di questa metrica con la \MetrLin.}; analizzando gli invarianti di questa metrica troviamo che:
$$
\left\{\eqalign{%
&det g=-\rho^4\sin^2\theta\cr
&R_{\mu\nu\alpha\beta}R^{\mu\nu\alpha\beta}={48M^2(a^6\cos^6\theta-15a^4r^2\cos^4\theta+15a^2r^4\cos^2\theta-r^6)\over\rho^6}
}\right.
$$
e quindi $\rho^2=0$ \`e la vera singolarit\`a della metrica in esame, in particolare per i valori di coordinate $r=0$ e $\theta=\pi/2$. Se analizziamo il limite per $M\to 0$ otteniamo la metrica Minkoskiana in coordinate ``ellittiche schiacciate'' che ci inducono a pensare che questa in realt\`a sia una singolarit\`a ad anello.
\par
Le altre singolarit\`a della metrica, cio\`e i valori per cui sia $g_{tt}$ e $\Delta$ si annullano e divergono rispettivamente, definiscono quattro superfici tramite le seguenti relazioni:
$$
\cases{%
r_\pm=M\pm\sqrt{M^2-a^2\cos^2\theta}&$\triangleleft$\ \hbox{Superficie di Stazionariet\`a}\cr
\hat{r}_{\pm}=M\pm\sqrt{M^2-a^2}&$\triangleleft$\ \hbox{Orizzonte degli eventi}
}
$$
dove per\`o solo quelle con il segno positivo definiscono delle superfici ``fisiche'' come segnalato in \Biblio{Cart68}{B.Carter}{Global Structure of the Kerr Family of Gravitational Fields}{Phys.Rev. 174,5(1968)}; per scoprire la particolarit\`a di queste bisogna studiare le curve di moto delle particelle di prova in questa metrica: prima di tutto si osserva che $g_{tt}$ cambia segno all'attraversamento di $r_+$ causando l'impossibilit\`a per una particella fisica di avere una curva del moto descritta da $r, \theta$ e $\phi$ costanti in quanto si avrebbe $\Vec{u}^2>0$; unico altro termine negativo in questa metrica \`e quello relativo a $g_{t\phi}$ che dunque ci indica il fatto che all'interno di questa regione si \`e obbligati ad avere un moto angolare e dunque non esistono osservatori stazionari rispetto alle coordinate all'infinito spaziale.
\par 
Studiando pi\'u in dettaglio le equazioni del moto possiamo isolare quattro integrali primi del moto:
\bigskip
\item{$\bullet$}{\bf Energia}\ $p_t=m(g_{tt}{dt\over ds}+g_{\phi t}{d\phi\over ds})$
\item{$\bullet$}{\bf Momento Angolare}\ $p_\phi=m(g_{\phi t}{dt\over ds}+g_{\phi\phi}{d\phi\over ds})$
\item{$\bullet$}{\bf On shell}\ $g_{\mu\nu}p^\mu p^\nu=-m^2$
\item{$\bullet$}{\bf Quarta costante di Carter}\Foot{Questa costante del moto \`e ``fortunata'' nel senso che si ottiene usando le equazioni di Hamilton-Jacobi e rappresenta in pratica il momento associato alla coordinata $\theta$.}\ ${\cal Q}=p^2_\theta+\left(ap_t\sin\theta-{p_\phi\over\sin\theta}\right)^2+a^2m^2\cos^2\theta$
\bigskip\noindent
che permettono di integrare completamente il moto; analizzando il termine $p_{\phi}$ possiamo riscriverlo come 
$$
\eqalign{%
p_{\phi}&=g_{t\phi}\dot{t}+g_{\phi\phi}\dot{\phi}\cr
&=g_{t\phi}\dot{t}+g_{\phi\phi}{d\phi\over dt}\dot{t}\cr
&=g_{\phi\phi}\dot{t}(\Omega-\omega)
}
$$
dove $\omega={d\phi\over dt}$ \`e la velocit\`a angolare misurata dagli osservatori esterni all'infinito spaziale e $\Omega=-{g_{t\phi}\over g_{\phi\phi}}$ \`e chiaramente una quantit\`a propria dell'ergosfera in quanto costruita a partire dalla metrica; nel punto in cui $\Delta=0$ si pu\`o assegnare il suo  valore essere la \vip{velocit\`a angolare dell'orizzonte}\Foot{Il calcolo pu\`o essere semplificato ponendo $g_{\phi \phi}={{\cal A}\over \rho^2}\sin^2\theta$ dove ${\cal A}=(r^2+a^2)^2-a^2\Delta \sin^2\theta$, sull'orizzonte $\Delta=0$.} 
$$
\Omega_H={a\over \hat r^2_++a^2}
$$ Per capire il significato di $\Omega$ prendiamo il caso di una particella avente velocit\`a angolare nulla all'infinito, siccome avvicinandosi al corpo rotante $\Omega\neq 0$ anche $\omega$ incrementer\`a il suo valore portando il corpo a ruotare solidale con l'oggetto stellare; corpi rientranti in questa casistica sono quelli che si avvicinano di pi\'u a osservatori inerziali (d'ora in poi li chiameremo \vip{ZAMO}\Foot{Zero angular momentum observer.}). Per loro il quadrivettore velocit\`a pu\`o essere scritto come
$$
\eqalign{%
u_Z&=u^t e_t+u^\phi e_{\phi}\cr
&=\gamma(e_t+\Omega\,e_\phi)
};
$$
Ridefinendo una tetrade rispetto a questi osservatori otteniamo
$$
\left\{\eqalign{%
&e_{\hat t}=\gamma(e_t+\Omega e_\phi)\cr
&e_{\hat r}={e_r\over\sqrt{g_{rr}}}\cr
&e_{\hat \theta}={e_\theta\over\sqrt{g_{\theta\theta}}}\cr
&e_{\hat \phi}={e_\phi\over\sqrt{g_{\phi\phi}}}\cr
}\right.
$$
Cerchiamo adesso la condizione sul moto di una ipotetica particella nell'ergosfera descritta rispetto a questa base affinch\'e l'energia misurata all'infinito sia negativa\Foot{Bisogna sempre ricordarsi che le coordinate che stiamo usando sono quelle di osservatori all'infinito, puntualmente gli ZAMO osserveranno sempre e solo energie positive (bisogna avere sempre presente il principio di equivalenza).}:
$$
\eqalign{
E_Z&=-p\cdot e_{\hat t}\cr
&=\gamma\left[-p\cdot e_t-\Omega p\cdot e_\phi\right]\cr
&=\gamma\left[E^\infty-\Omega p_\phi\right]
}
$$
Dunque si deve avere $E^\infty>\Omega p_\phi$ il che permette ad una particella proveniente dall'in\-fi\-ni\-to con un energia positiva di separarsi in una porzione ad energia negativa (sempre misurata all'in\-fi\-ni\-to) che in seguito cadr\`a nell'\Oe\ ed un'altra por\-zio\-ne che a questo punto fuggir\`a di nuovo all'infinito con un energia maggiore di quella posseduta all'inizio del processo: questo processo viene chiamato \vip{processo di Penrose}\ e permette di acquistare energia a spese del campo gravitazionale di un buco nero. Questa variazione sar\`a vincolata dalla formula precedentemente trovata ad essere
$$
\delta M=\Omega_H\delta J
$$
cio\`e la massa del buco nero aumenter\`a ma di conseguenza diminuir\`a il suo momento angolare (e di conseguenza diminuir\`a l'ampiezza dell'ergosfera); tenedo conto della loro forma funzionale e integrando si ottiene che la massa di un buco nero ha un contributo dal momento angolare dato dalla formula
$$
M^2=M^2_{Ir}+{J^2\over4M^2_{Ir}}\Eqno
$$\COUNT{MassaIrr}
dove $M_{Ir}={1\over2}(\hat r_+^2+a^2)^{1/2}$ \`e la costante di integrazione corrispondente a $J=0$ ed in questo caso rappresenta la quantit\`a di energia non estraibile da un buco nero.
\bigskip
\Figura{ergosfera}{6cm}{6cm}{In questa figura sono riassunte le varie zone di interesse fisico di un buco nero rotante.}
Bench\'e oggi si prediliga la nozione di entropia come mancanza di informazione (Shannon) non pu\`o sfuggire l'analogia con l'entropia dei gas come \`e stata definita originariamente. In principio l'entropia di un gas \`e stata definita per misurare quanta parte dell'energia contenuta nel gas sotto forma di calore pu\`o essere riconvertita in lavoro e quanta invece sar\`a destinata a rimanere calore.\par
La presenza di un limite nelle possibilit\`a di estrazione di energia da un buco nero \`e forse stata la prima indicazione dlle possibilit\`a di associare un'entropia ai buchi neri.
%
%
%
\SottoSezione{Kerr Newman}
Soluzione trovata da Newman et al. nel 1965 che scritta in coordinate sferiche risulta
$$
g=-\left(\Delta\Sigma\over{\cal A}\right)\Tensor{dt}+{{\cal A}\sin^2\theta\over\Sigma}\Tensor{(d\phi-\omega dt)}+\left(\Sigma\over\Delta\right)\Tensor{dr}+\Sigma\Tensor{d\theta}
$$
dove
$$
\eqalign{
&\Sigma=r^2+a^2\cos^2\theta\cr
&\omega={a\over {\cal A}}(2mr-Q^2)\cr
&\Delta=r^2+a^2-2mr+Q^2\cr
&{\cal A}=(r^2+a^2)^2-a^2\Delta\sin^2\theta
}
$$
Non \`e una soluzione nel vuoto, in quanto si assume la presenza di un campo elettrico generato dalla $1$-forma
$$
{\bf A}=-{Qr\over\Sigma}({\bf dt}-a\sin^2\theta{\bf d\phi})
$$
Le singolarit\`a della metrica sono analoghe a quelle di Kerr: si ha una singolarit\`a ad anello per i valori di $r$ tali che $\Sigma=0$ con l'unica differenza che esiste l'orizzonte degli eventi solamente nel caso 
$$
M^2\geq a^2+Q^2
$$
Le particelle cariche seguiranno curve di moto descritte da
$$
u^\mu\nabla_\mu u^\nu=e\Mat{F}{\nu}{\alpha}u^\alpha
$$
ed il calcolo per il moto di particelle precipitate nel buco nero \`e analogo a quello della metrica di Kerr, portando ad una nuova relazione fra la massa $M$ e le grandezze $J$ e $Q$ data da
$$
M^2=\left(M_{Ir}+{Q^2\over4M_{Ir}}\right)^2+{J^2\over4M^2_{Ir}}
$$
L'importanza di questa famiglia di metriche a $3$ parametri sta nel fatto che \`e stato dimostrato che \`e l'unica classe di soluzione stazionarie a simmetria assiale, a\-sin\-to\-ti\-ca\-men\-te piatta che ammette orizzonte degli eventi: in pratica qualunque corpo isolato  che per cause incognite risulti collassare oltre il suo orizzonte sar\`a per forza descritto solo dalle grandezze $(M,J,Q)$.\Foot{Questa \`e ci\`o che viene riassunto dal famoso detto: {\sl A Black hole has no hair}, che \`e l'enunciato della cosidetta no-hair conjecture.}
%
%
%
\Section{Termodinamica dei Buchi Neri}
Nel 1970 Christodoulou \Biblio{Chri70}{D.\ Christodoulou}{Reversible and irreversible transformation in black hole physics}{Phys. Rev. Lett. 25,1596 (1970)} ricava la formula \MassaIrr\ studiando il processo di Penrose di estrazione di energia da un buco nero rotante ed introduce la massa irriducibile, grandezza che rappresenta il quantitativo di energia non estraibile da un buco nero oltre che quello di \vip{trasformazioni reversibili} aventi la caratteristica di lasciare invariato il valore di questa grandezza\Foot{Viceversa saranno chiamate irreversibili quelle che lo modificano.}; in seguito Bekenstein \Biblio{Beke73}{J.\ D.\ Bekenstein}{Black holes and entropy}{Phys. Rev. D 7,2333 (1973)}, come dichiara nel suo articolo, fa un tentativo di unificazione della termodinamica con la fisica dei buchi neri: l'area\Foot{Ovviamente ${\cal H}$ \`e in realta una ipersuperficie, qui si intende il suo fogliettamento bidimensionale. } dell'orizzonte ${\cal H}$ in una metrica di Kerr-Newman \`e data dalla relazione 
$$
A({\cal H})=4\pi(2M\hat r_+-Q^2)
$$
\Figura{DiagrammaPVT}{10cm}{7cm}{Diagramma dello spazio delle variabili ``termodinamiche'' del buco nero.} 
\noindent Differenziando e esplicitando $dM$ si ottiene la seguente relazione
$$
dM=\Theta\, d\alpha+\vec{\Omega}\cdot\,d\vec{J}+\Phi\, dQ\quad\hbox{dove}\quad\left\{
\eqalign{
&\Theta={1\over4}(\hat r_+-\hat r_-)/\alpha\cr
&\vec\Omega=\vec a/\alpha\cr
&\Phi=Q\hat r_+/\alpha
}\right.
$$
avente strettissima analogia formale con la conosciuta formula termodinamica
$$
dE=T\,dS-p\,dV
$$
interpretando $\vec{\Omega}\cdot\vec{J}+\Phi\, dQ$ come il lavoro fatto sul sistema, $\Theta$ con la temperatura del sistema ed infine $\alpha={A({\cal H})\over 4\pi}$ con l'entropia. Quest'ultima identificazione non pareva campata in aria siccome ogni trasformazione da una condizione di equilibrio ad un'altra nella metrica suddetta era sempre caratterizzata da una variazione positiva, o in casi particolari nulla, della cosidetta area razionalizzata $\alpha$, proprio come l'entropia dei sistemi termodinamici.\par
Ovviamente entropia e superficie hanno due unit\`a di misura differenti e dunque per definirla si appella alla meccanica quantistica usando la costante di Planck $(\,\hbar\,)$ e cerca di ricavare tramite la definizione statistica di entropia, cio\`e
$$
S=-\sum_np_n\ln p_n
$$
il valore minimo di variazione dell'area dell'\Oe\ in seguito alla cattura di una particella e lo identifica ad un bit di informazione persa grazie alla relazione
$$
\Delta I=-\Delta S
$$
trovando che \`e indipendente dai parametri del buco nero ma non da quelli della particella; in particolare per un corpo di raggio $b$ e massa $\mu$ si ha
$$
\Delta \alpha=2\mu b
$$
deducendo per una particella elementare l'aumento deve essere dell'ordine di gran\-dez\-za di $2\,\hbar$. Siccome un sistema fisico in genere possiede dell'entropia, quando questo cade oltre l'orizzonte, succede che l'entropia dell'universo fuori dal buco nero diminuisce, ma di conseguenza aumenta l'area dello stesso, quindi  enuncia quella che chiama la \vip{seconda legge generalizzata}: {\sl la somma dell'entropia all'esterno del buco nero pi\'u quella assegnata al buco nero non diminuisce in nessun caso}.
\Example{Motore Gravitazionale}
Si prenda un sistema costituito da un buco nero, una scatola di raggio $b$ contenente radiazione all'equilibrio ad una temperatura $T$ ed un sistema atto a calare questa scatola fino all'\Oe; si cali la scatola all'interno dell'ergosfera fino ad arrivare sulla superficie dell'\Oe\ cos\'\i\ che per  red-shift l'energia della scatola\Foot{Per ragguagli visionare l'appendice di [Beke73].} sia pari a $E=2\mu b\Theta$ e  a questo punto si rilasci la radiazione, facendo perdere cos\'\i\ un quantitativo pari a $\Delta\mu$ di energia oltre l'\Oe. Si  riporti nella configurazione iniziale il tutto ottenendo cos\'\i\ un quantitativo di lavoro (fatto dal sistema) pari a $\Delta\mu(1-2b\Theta)$: come motore termico esso ha una efficienza pari a $\epsilon=1-2b\Theta$! Considerando che per avere raggio $b$ e contenere della radiazione senza farla sfuggire, deve esserci un vincolo sulla temperatura del ``box'' che porta l'efficienza ad avere una forma del tipo
$$
\epsilon<1-{T_{Bh}\over T}
$$
proprio come un motore di Carnot gravitazionale.
FineEx
Tuttavia la seconda legge sembra cadere nel caso venga ``immerso'' il nostro buco nero in un bagno termico ad una temperatura $T_{Ex}$ minore di quella formale del buco nero: infatti si avrebbe
$$
\eqalign{
d(S_{Bh}+S_{Ex})&={\partial\over \partial M}(S_{Bh}+S_{Ex})\,dM\cr
&=\left({1\over T_{Bh}}-{1\over T_{Ex}}\right)\,dM\cr
&\leq 0
}
$$
cos\'\i\ da violare la Seconda legge Generalizzata! Vedremo in seguito come Hawking risolva in parte questo problema grazie alla meccanica quantistica.


