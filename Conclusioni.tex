\Titolone{Conclusioni}{\hfill``Tutto va imparato non per esibirlo ma per adoperarlo''}{De Vauvenargues}
Alla fine di questo  lavoro  si \`e giunti alla conclusione che l'entropia associabile ad una cosidetta singolarit\`a nuda non solo rimane costante, ma addirittura pu\`o essere posta a zero in accordo con una interpretazione informazionale di questa grandezza.
\par
Tuttavia a questo lavoro manca per essere completo di quella teoria quantistica della gravit\`a che si pone di individuare almeno nei tratti caratteristici: nel calcolo della soluzione di collasso di polveri per esempio, non si \`e tenuto conto degli effetti quantistici inevitabilmente presenti quando la curvatura (o qualunque altra grandezza caratterizzante la teoria gravitazionale di Einstein) tende ad un valore illimitato\Foot{O comunque nel caso si cerchi di trattare problemi all'ordine di grandezza delle unit\`a di \vip{Planck}, quantit\`a che si possono formare usando le costanti fondamentali $G,\,\hbar$ e $c$; esse hanno un valore fuori dalle possibilit\`a degli attuali strumenti scientifici.}.
\par
Gi\`a negli esempi trattati, in cui si applicava la teoria dei campi ad un background metrico, si \`e trovato il cosidetto fenomeno dell``evaporazione di buchi neri'': in particolare con diverse metodologie, sia concettuali che di calcolo, si \`e arrivati a stabilire l'emisione di particelle verso osservatori esterni nel caso di comportamenti singolari della metrica. Questa radiazione pone un termine aggiuntivo al tensore energia-impulso nel membro destro della \EqEi, termine che nel caso della formazione della singolarit\`a nuda risulta addirittura infinito, pregiudicando previsioni accettabili di questa teoria rimanendo in ambito classico.
\par
Inoltre bisogna tenere conto del fatto che per portare fino in fondo l'analogia della fisica dei buchi neri con la termodinamica, bisognerebbe trattare sistemi stazionari e non dinamici, in quanto la termodinamica \`e capace di previsioni attendibili solo nel caso di sistemi all'equilibrio; anche questo pone dei limiti sullo studio di queste singolarit\`a metriche in quanto non pare esistano soluzioni accettabili dal punto di vista fisico in cui queste permangano per un tempo illimitato senza essere coperte da un orizzonte.
\par
Pur con tutte queste limitazione si spera di poter porre questo lavoro, ed in particolare il risultato ottenuto, come punto di partenza per eventuali approfondimenti sulla natura di questi bizzarri oggetti geometrici.
